\documentclass[BCOR10mm,headinclude,fontsize=12,headsepline,numbers=endperiod]{scrbook}

\usepackage{fontspec}

\usepackage{reledmac}

\firstlinenum{0} \linenumincrement{1}

\Xnumberonlyfirstinline[A]
\Xinplaceofnumber[A]{26mm}

\Xtwolines{f.}
\Xtwolinesbutnotmore

\Xhangindent{2,6cm}

\Xboxlinenum{2,6cm}

\Xboxstartlinenum{1.3em}
\Xboxendlinenum{0.5em}
\Xafternumber{0em}

\Xsublinesep{}
\sublinenumberstyle{alph}

\begin{document}\setlength{\parindent}{0pt}

\beginnumbering

\leftskip=2,6cm
\rightskip=1cm

\pstart Head: Subline number position\pend\vspace{0.75em}
\pstart\leftskip=0cm\edtext{\raisebox{-.25\baselineskip}[\ht\strutbox]{\parbox{2,6cm} {\small\textit{Hand-}}}This text was written using an old school typewriter. Only line\linebreak\raisebox{-.25\baselineskip}[\ht\strutbox]{\parbox{2,6cm} {\small\textit{written}}}numbers of this text are printed in the margin of the main sec-\linebreak
\raisebox{-.25\baselineskip}[\ht\strutbox]{\parbox{2,6cm} {\small\textit{Addition}}}}{\lemma{\textit{Handwritten Addition}}\linenum{||1|||1}\Afootnote{added manually}}tion of this page. Therefore, subline numbers are only printed in\linebreak\parbox{2,6cm}{{\ }}the footnotes.\pend \vspace{0.4em}\pstart There is no subline seperator, only a \edtext{symbol}{\lemma{symbol}\Afootnote{typewritten correction of: sombyl}} attached without spacing to the full line number ("1a", e.g.). This symbol should not alter the position of the (sub-)line number, that is the digits should remain exactly above (and/or below) the digits of \edtext{footnotes}{\lemma{footnotes}\Afootnote{handwritten correction of: rootnotes}} refering to full lines (contrary to what it looks now). That means that a subline number in the footnotes should extend into the space between the numbers and the lemma, similar to what a range symbol like "f." or a dash followed by an endnumber do, when Xboxstartlinenum and Xboxendlinenum are enabled. \pend\vspace{0.4em}\pstart\rightskip=0cm There is yet another paragraph where alterations occur, \edtext{too}{\lemma{too}\Afootnote{typewritten correction of: two}}, so \edtext{\raisebox{.5\baselineskip}[\ht\strutbox]{\parbox{1cm} {\small{[?]}}}}{\lemma{[?]}\linenum{||1|||1}\Afootnote{unidentified handwritten addition}}\linebreak there will be footnotes in full lines and sublines for two-digit-\parbox{1cm}{{\ }}\linebreak lines as well.
\pend

\endnumbering

\end{document}

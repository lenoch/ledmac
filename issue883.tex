\documentclass{book}
\listfiles
\usepackage{fontspec}
\usepackage{polyglossia}
\setmainlanguage{french}
\usepackage[]{setspace}
\usepackage[series={A},nocritical,noend,noeledsec,noledgroup]{reledmac}
\usepackage[shiftedpstarts]{reledpar}
\maxhnotesX{0.5\textheight}

\usepackage{lipsum}


\begin{document}

\begin{pages}
\begin{Leftside}


\beginnumbering

\setstanzaindents{5,5,5,5,5,5,5,5}
\stanza[\section*{Gedichte}\subsection*{Röslein auf der Heiden}]
Sah' ein Knab ein Röslein stehn\footnoteAnomk{\lipsum[1]}&
Röslein auf der Heiden\footnoteAnomk{\lipsum[1]},&
War so jung und morgenschön\footnoteAnomk{\lipsum[1]},&
Lief er schnell, es nah zu sehn\footnoteAnomk{\lipsum[1]},&
Sah's mit vielen Freuden\footnoteAnomk{\lipsum[1]}.&
Röslein, Röslein, Röslein rot\footnoteAnomk{\lipsum[1]},&
Röslein auf der Heiden\footnoteAnomk{\lipsum[1]}\&

\endnumbering

\end{Leftside}

\begin{Rightside}

\beginnumbering


\setstanzaindents{5,5,5,5,5,5,5,5}
\stanza[\section*{Poèmes}\subsection*{Petite rose sur la lande}]
Un enfant vit une petite rose\footnoteAmk &
Petite rose sur la lande\footnoteAmk &
Si jeune et belle en ce matin\footnoteAmk &
Qu'il accourut pour la voir de près \footnoteAmk
&
Il la vit avec joie\footnoteAmk&
Rose, rose, rose rouge\footnoteAmk
&
Petite rose sur la lande\footnoteAmk\&

\endnumbering
\end{Rightside}
\end{pages}
\Pages





\end{document}

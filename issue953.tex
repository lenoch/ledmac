\documentclass[b5paper]{book}

\usepackage[noledgroup,noeledsec,nocritical,series={},nofamiliar]{reledmac}
\lineation{page}

\begin{document}
\beginnumbering
\numberpstarttrue
\pstart\edlabel{4}%
Quel ramo del lago di Como, che volge a mezzogiorno, tra due catene non interrotte di monti, tutto a seni e a golfi, a seconda dello sporgere e del rientrare di quelli, vien, quasi a un tratto, a ristringersi, e a prender corso e figura di fiume, tra un promontorio a destra, e un’ampia costiera dall’altra parte; e il ponte, che ivi congiunge le due rive, par che renda ancor più sensibile all’occhio questa trasformazione, e segni il punto in cui il lago cessa, e l’Adda rincomincia, per ripigliar poi nome di lago dove le rive, allontanandosi di nuovo, lascian l’acqua distendersi e rallentarsi in nuovi golfi e in nuovi seni.
\pend
\textbf{4 : \pstartref{4}}
\pstart\edlabel{4bis}%
Quel ramo del lago di Como...
\textbf{4 bis: \pstartref{4bis}}
\begin{edtabularl}
    1 & 2 & 3\\
    1 & 2 & 3\\
    1 & 2 & 3
\end{edtabularl}
\pend

\pstart\edlabel{5}%
Quel ramo del lago di Como, che volge a mezzogiorno, tra due catene non interrotte di monti, tutto a seni e a golfi, a seconda dello sporgere e del rientrare di quelli, vien, quasi a un tratto, a ristringersi, e a prender corso e figura di fiume, tra un promontorio a destra, e un’ampia costiera dall’altra parte; e il ponte, che ivi congiunge le due rive, par che renda ancor più sensibile all’occhio questa trasformazione, e segni il punto in cui il lago cessa, e l’Adda rincomincia, per ripigliar poi nome di lago dove le rive, allontanandosi di nuovo, lascian l’acqua distendersi e rallentarsi in nuovi golfi e in nuovi seni.
\pend
\textbf{5 : \pstartref{5}}
\endnumbering
\end{document}

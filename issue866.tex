\input regression-test
\showoutput
\documentclass{book}
%Edizione Critica
\usepackage[nofamiliar,noend,noeledsec,noledgroup,series={C}]{reledmac}
\labelpstarttrue
\lineation{page}
\linenummargin{inner}
\Xnotefontsize[C]{\footnotesize}
\Xpstart[C]
\Xpstartonlyfirst[C]
\Xstanza[C]
\Xstanzaonlyfirst[C]

\makeatletter
\renewcommand{\thepstart}{\textbf{\arabic{pstart}}}
\makeatother

\setcounter{stanzaindentsrepetition}{2}
\setstanzaindents{8,0,1}
\begin{document} \makeatletter \let\@bidi@pdfcustomproperties\relax \makeatother             Font initialisation\START
\beginnumbering
\numberpstarttrue

s\vskip0.9\textheight
\pstart%
Quel ramo del lago di Como che volge a mezzogiorno
\pend

\pstart
e il \edtext{ponte}{\Cfootnote{via}}, che ivi \edtext{ponte}{\Cfootnote{via}} congiunge le due rive, par che renda ancor più sensibile all'occhio questa trasformazione, e segni il punto in cui il lago cessa, e l'Adda rincomincia, per ripigliar poi nome di \edtext{lago}{\Cfootnote{fiume}} dove le rive, allontanandosi di nuovo, lascian l'acqua distendersi e rallentarsi in nuovi golfi e in nuovi seni
\pend


\numberpstartfalse
\setcounter{stanza}{2}
\numberstanzatrue
\stanza
\edlabel{begin:1}\edtext{Lorem}{\lemma{Lorem\ldots nisis}\xxref{begin:1}{end:1}\Cfootnote{A note on two verses}} ipsum dolor sit amet, consectetur adipisicing elit,&
sed do eiusmod tempor incididunt ut labore et dolore&
magna aliqua. Ut enim ad minim veniam, quis nostrud&
exercitation ullamco laboris nisi\edlabel{end:1}&
\edtext{ut aliquip}{\Cfootnote{ut aliliquip}} consequat ut aliquip consequat irure dolor in reprehenderit irure dolor in reprehenderit&
 irure dolor in reprehenderit&
in voluptate velit esse cillum dolore eu ur. Excepteur sint occaecat&
cupidatat non proident, sunt in culpa qui officia deserunt&
\edlabel{begin:2}\edtext{Duis}{\xxref{begin:2}{end:2}\lemma{Duis\ldots occaecat}\Cfootnote{Another note on two verses}} aute irure dolor in reprehenderit&
in voluptate velit esse cillum dolore eu fugiat nulla&
pariatur. Excepteur sint occaecat\edlabel{end:2}\&

\numberstanzafalse
\endnumbering
\end{document}

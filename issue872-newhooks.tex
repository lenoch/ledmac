\documentclass{article}
\usepackage[main=latin]{babel}

\usepackage[series={A,B},nofamiliar,noeledsec,noledgroup]{reledmac}

\sethangingsymbol{[\,}
\firstlinenum{1}
\linenumincrement{1}
\setstanzaindents{1,0,0}
\setcounter{stanzaindentsrepetition}{1}
\numberstanzatrue
\Xstanza
\newcommand{\Abothnote}[1]{\Afootnote{#1}\Aendnote{#1}}
\newcommand{\Bbothnote}[1]{\Bfootnote{#1}\Bendnote{#1}}
\begin{document}
\beginnumbering

\stanza
\edlabel{begin:1}\edtext{Lorem}{\lemma{Lorem\ldots nisis}\xxref{begin:1}{end:1}\Abothnote{A note on four lines of verse, the first of which was missing in the first edition}} ipsum dolor sit amet, consectetur adipisicing elit,&
\linenumannotation{a}sed do eiusmod tempor incididunt ut labore et dolore&
\linenumannotation{b}magna aliqua. Ut enim ad minim veniam, quis nostrud&
\linenumannotation{c}exercitation ullamco laboris nisi\edlabel{end:1}&
\linenumannotation{d}\edtext{ut aliquip}{\Abothnote{ut aliliquip}} consequat ut aliquip consequat irure dolor in reprehenderit irure dolor in reprehenderit&
\linenumannotation{e}\edtext{Duis aute}{\Bbothnote{Some comments}} irure dolor in reprehenderit&
\linenumannotation{f}\edlabel{begin:2}\edtext{in}{\xxref{begin:2}{end:2}\lemma{in\ldots deserunt}\Abothnote{Theses two lines of verse were one single line in the first edition}} voluptate velit esse cillum dolore eu ur. Excepteur sint occaecat&
\linenumannotation{f}cupidatat non proident, sunt in culpa qui officia deserunt\edlabel{end:2}&
\linenumannotation{g}\edlabel{begin:3}\edtext{Duis}{\xxref{begin:3}{end:3}\lemma{Duis\ldots occaecat}\Abothnote{Another note on two verses}} aute irure dolor in reprehenderit&
\linenumannotation{g}in voluptate velit esse cillum dolore eu fugiat nulla&
\linenumannotation{h}pariatur. Excepteur sint occaecat\edlabel{end:3}&
\edtext{A line with no annotation}{\Abothnote{indeed, there is not, but for A series, there is line number}\Bbothnote{indeed, there is not, but for B series, no line number}}
\&
\endnumbering

\bigskip
\doendnotes{A}

\doendnotes{B}

\end{document}

\documentclass{article}
\usepackage[osf,nomath,p]{libertinus}
\usepackage{microtype}
\usepackage[pdfusetitle,hidelinks]{hyperref}

\usepackage[series={},nocritical,noend,nofamiliar,noledgroup]{reledmac}
\usepackage{reledpar}

\newfontfamily\syriacfont{Estrangelo Edessa}[Script=Syriac,Scale=1.2]

\usepackage{ifluatex}
\ifluatex
  \newcommand{\textsyriac}[1] % Syriac inside LTR
             {\bgroup\textdir TRT\syriacfont #1\egroup}
  \newenvironment{syriac}     % Syriac paragraph
             {\textdir TRT\pardir TRT\syriacfont}{}
\else
  \usepackage{graphicx}
  \usepackage{polyglossia}
  \setmainlanguage{english}
  \setotherlanguage{syriac}
  \renewcommand{\textsyriac}[1]{\bgroup\syriacfont #1\egroup}
\fi
\usepackage{metalogo}

\newcommand{\n}[1] % for digits inside Arabic text
           {\bgroup\LTR  #1\egroup}
\newcommand{\syriacfootnote}[1] % Syriac Footnotes
           {\footnote{\textsyriac{#1}}}

\begin{document}
\date{}
\title{Using \LuaLaTeX\ to typeset RTL text with reledpar}
\maketitle
\begin{abstract}
This file provides an example of how to use reledpar and \LuaLaTeX\ to typeset a right to left text with its translation on the facing page\footnote{The text was provided by Latechneuse at \url{https://tex.stackexchange.com/q/227837/}.}.  

As you can see, the switch to RTL convention is made \emph{before} the \verb+pstart+.
It must be also called inside \verb+\eledsection+.

For an example with \XeLaTeX, look at \href{./parallel-column-two-languages.tex}{parallel-column-two-languages.tex} file.
\end{abstract}

\begin{pages}
\begin{Leftside}
\begin{syriac}
\beginnumbering
   \pstart
       \eledsection*{\textsyriac{ܡܿܟܪܟܝ}}
   \pend

   \pstart
        1ܘܟܕ 2ܡܿܟܪܟܝ3ܢܢ ܐܪܟ4ܐܢܐ ܗ̄ 5ܡܘܪܐ6 ܗܿܝ ܩ7ܕܡܝܬܐ
   \pend
\endnumbering
\end{syriac}
\end{Leftside}

\begin{Rightside}
\beginnumbering
   \pstart
       \eledsection{English headline} 
   \pend

   \pstart
        Some english text. 
   \pend
\endnumbering
\end{Rightside}

\end{pages} 
\Pages
\end{document}

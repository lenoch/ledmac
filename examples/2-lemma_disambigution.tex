\documentclass{article}
\usepackage[T1]{fontenc}
\usepackage[osf,p]{libertinus}
\usepackage{microtype}
\usepackage[pdfusetitle,hidelinks]{hyperref}
\usepackage[english, main=latin]{babel}
\babeltags{english = english}

\usepackage[series={A,B},nofamiliar,noend,noeledsec,noledgroup]{reledmac}
\begin{document}


\begin{english}
\date{}
\title{Lemma's disambiguation}
\maketitle

\firstlinenum{1}
\linenumincrement{1}

\begin{abstract}
This file provides an example of lemma's disambuigation.

All word which can potentially be twice (or more) in a same line is marked by an \verb+\sameword+. \emph{reledmac} prints the word rank only if the word is effectively printed twice (or more) time in the same line.

For use with \verb+\lemma+, please read the handbook.

\end{abstract}
\end{english}





\beginnumbering
\pstart
Leo \sameword{aut} ursus \sameword{aut} oryx \sameword{aut} ricinus \sameword{aut} equus \sameword{aut}
lupus \edtext{\sameword{aut}}{\Afootnote{et}\Bfootnote{monotone\ldots}} canis \sameword{aut} felix \sameword{aut} asinus \edtext{\sameword{aut}}{\Afootnote{et}} burricus.

\pend
\endnumbering







\end{document}

\documentclass{article}
\usepackage[T1]{fontenc}
\usepackage[osf,p]{libertinus}
\usepackage{microtype}
\usepackage[pdfusetitle,hidelinks]{hyperref}
\usepackage[english, main=latin]{babel}
\babeltags{english = english}

\usepackage{setspace}
\AtBeginDocument{\doublespacing}
\usepackage{lipsum}
\SetLipsumParListEnd{\par}
\usepackage[series={A,B},noend,noeledsec,nofamiliar,noledgroup]{reledmac}
\Xarrangement[A]{paragraph}
\Xarrangement[B]{twocol}
\makeatletter
   \Xbhooknote{\setstretch {\setspace@singlespace}}
   \Xbhookgroup{\setstretch {\setspace@singlespace}}
\makeatother


\begin{document}

\begin{english}
\date{}
\title{Line spacing}
\maketitle
\begin{abstract}
This example sets the interline of footnotes with the \emph{setspace} package. The main text is double line spacing, the footnotes are single line spacing.

It use the \verb+\Xbhooknote+ and \verb+\Xbhookgroup+ commands to call some space setting.
Note that these commands are called after loading the \emph{setspace} package, because they use \emph{setspace} commands.

We use \verb`\AtBeginDocument{\doublespacing}` to start doublespacing at the beginning of the \verb+document+ environment. We do not use the \verb+doublespacing+ option of  the \emph{setspace} package, because it dirupts internal \emph{reledmac} computing.
\end{abstract}
\end{english}
\beginnumbering
\pstart
\lipsum*[1]\edtext{Lipsum}{
    \Afootnote{\lipsum*[3]}
    \Bfootnote{\lipsum*[4]}}
\lipsum*[6-7]
\pend
\endnumbering
\end{document}

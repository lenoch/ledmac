\documentclass[10pt]{article}

\usepackage{reledmac}
\linenummargin{right}
\frenchspacing

\usepackage{hyperref}

\begin{document}
  \beginnumbering
  \pstart[\section{Tac. \emph{Ann.} 1.12}]
Inter quae senatu ad infimas obtestationes procumbente, dixit forte Tiberius se ut non toti rei publicae parem, ita quaecumque pars sibi mandaretur eius tutelam suscepturum. tum Asinius Gallus 'interrogo' inquit, 'Caesar, quam partem rei publicae mandari tibi velis.' perculsus inprovisa interrogatione paulum reticuit: dein collecto animo respondit nequaquam decorum pudori suo legere aliquid aut evitare ex eo cui in universum excusari mallet. rursum Gallus (etenim vultu offensionem coniectaverat) non idcirco interrogatum ait, ut divideret quae separari nequirent sed ut sua confessione argueretur unum esse rei publicae corpus atque unius animo regendum. addidit laudem de Augusto Tiberiumque ipsum victoriarum suarum quaeque in toga per tot annos egregie fecisset admonuit. nec ideo iram eius lenivit, pridem invisus, tamquam ducta in matrimonium Vipsania M. Agrippae filia, quae quondam Tiberii uxor fuerat, plus quam civilia agitaret Pollionisque Asinii patris ferociam retineret.
  \pend

  \pstart[\section{Tac. \emph{Ann.} 1.13}]
Post quae L. Arruntius haud multum discrepans a Galli oratione perinde offendit, quamquam Tiberio nulla vetus in Arruntium ira: sed divitem, promptum, artibus egregiis et pari fama publice, suspectabat. quippe Augustus supremis sermonibus cum tractaret quinam adipisci principem locum suffecturi abnuerent aut inpares vellent vel idem possent cuperentque, M'. Lepidum dixerat capacem sed aspernantem, Gallum Asinium avidum et minorem, L. Arruntium non indignum et si casus daretur ausurum. de prioribus consentitur, pro Arruntio quidam Cn. Pisonem tradidere; omnesque praeter Lepidum variis mox criminibus struente Tiberio circumventi sunt. etiam Q. Haterius et Mamercus Scaurus suspicacem animum perstrinxere, Haterius cum dixisset 'quo usque patieris, Caesar, non adesse caput rei publicae?' Scaurus quia dixerat spem esse ex eo non inritas fore senatus preces quod relationi consulum iure tribuniciae potestatis non intercessisset. in Haterium statim invectus est; Scaurum, cui inplacabilius irascebatur, silentio tramisit. fessusque clamore omnium, expostulatione singulorum flexit paulatim, non ut fateretur suscipi a se imperium, sed ut negare et rogari desineret. constat Haterium, cum deprecandi causa Palatium introisset ambulantisque Tiberii genua advolveretur, prope a militibus interfectum quia Tiberius casu an manibus eius inpeditus prociderat. neque tamen periculo talis viri mitigatus est, donec Haterius Augustam oraret eiusque curatissimis precibus protegeretur.
  \pend

  \pstart[\section{Tac. \emph{Ann.} 1.19}]
Aggerabatur nihilo minus caespes iamque pectori usque adcreverat, cum tandem pervicacia victi inceptum omisere. Blaesus multa dicendi arte non per seditionem et turbas desideria militum ad Caesarem ferenda ait, neque veteres ab imperatoribus priscis neque ipsos a divo Augusto tam nova petivisse; et \edtext{parum in tempore}{\Aendnote{See \emph{OLD}, \textit{s.v.} tempus 8d.}} incipientis principis curas onerari. si tamen tenderent in pace temptare quae ne civilium quidem bellorum victores expostulaverint, cur contra morem obsequii, contra fas disciplinae vim meditentur? decernerent legatos seque coram mandata darent. adclamavere ut filius Blaesi tribunus legatione ea fungeretur peteretque militibus missionem ab sedecim annis: cetera mandaturos ubi prima provenissent. profecto iuvene modicum otium: sed superbire miles quod filius legati orator publicae causae satis ostenderet necessitate expressa quae per modestiam non obtinuissent.
  \pend
  \endnumbering

  \section{Notes}
  \doendnotes{A}

\end{document}
